\documentclass{article}
\usepackage{graphicx}
\usepackage{listings}
\usepackage{amsmath}
\begin{document}

\section*{3.1-1} 

Let $f(n)$ and $g(n)$ be asymptotically nonnegative functions. Using the basic definition of $\Theta$-notation, prove that $max(f(n), g(n)) = \Theta(f(n)+g(n))$

if $f(n)$ and $g(n)$ be asymptotically nonnegative functions:

\[ f(n) \le max(f(n), g(n)) \]
\[ g(n) \le max(f(n), g(n)) \]
\[ f(n) + g(n)  \le 2*max(f(n), g(n)) \]
\[ 1/2(f(n) + g(n))  \le max(f(n), g(n)) \le f(n) + g(n)\], 
so $max(f(n), g(n)) = \Theta(f(n)+g(n))$ 

\section*{3.1-2}

Show that for any real constants $a$ and $b$, where $b>0$,
\[ (n+a)^b = \Theta(n^b) \]
 
The fastest growing part is $n^b$, so $c_1*n^b \ge (n+a)^b \ge c_2*n^b$, for some $c_1$ and $c_2=a^b$. 
So,
\[ (n+a)^b = \Theta(n^b) \]

\section*{3.1-3}

Explane why the statement, "The running time of algorithm A is at least $O(n^2)$," is meaningless.

O- show the upper bound of time. "At least" means lower bound. This concepts contradict each other.

\section*{3.1-4}

Is $2^{n+1}=O(2^n)$? Is $2^{2n} = O(2^n)$?

\[ 2^{n+1} = 2*2^n \], so 
\[ 2*2^n \le c*2^n\], for $c \ge 2$
So  $2^{n+1}=O(2^n)$


\[ 2^{2n} = 2^n*2^n \]
\[ 2^n*2^n \le c*2^n \]

$2^n$ is unbounded function, so there is no c bigger than $2^n$ for all n. So  $2^{2n} \ne O(2^n)$

\section*{3.1-5}

Prove Theorem 3.1.

For any two function $f(n)$ and $g(n)$, we have $f(n)=\Theta(g(n))$ if and only if $f(n)=O(g(n))$ and $f(n)=\Omega(g(n))$.

if $g(n)$ is upper bound and lower bound function than $g(n)$ is tight bound. So if $f(n)=O(g(n))$ and $f(n)=\Omega(g(n))$ then $f(n)=\Theta(g(n))$ .

Let us assume $f(n)=\Theta(g(n))$ and there is same $n_k$ $f(n_k)>O(g(n_k))$. But in this case  $\Theta(g(n))$ is not a tight bound. Contradiction.
Same with $f(n_k)<\Omega(g(n_k))$

\section*{3.1-6}
Prove that the running time of an algorithm is $\Theta(g(n))$ if and only if its worst-case running time is  $O(g(n))$ and its best-case running time is  $\Omega(g(n))$

From Theorem 3.1

\section*{3.1-7}
Prove that $o(g(n)) \cap \omega(g(n))$ is the empty set.

\[ f(n) = o(g(n)) : c*g(n) >f(n) \] , there is no c that $c*g(n) \le f(n)$
\[ f(n) = \omega(g(n)) : c*g(n) < f(n) \], there is no c that $c*g(n) \ge f(n)$ 

Contradiction.

\section*{3.1-8}

We can extend out notation to the case of two parameters $n$ and $m$ that can go to infinity independently at different rates. For a given function $g(n,m)$, we denote by $O(g(n,m))$ the set of functions


\[ O(g(n,m)) = 
\begin{cases}
       \text{f(n,m),} &\quad\text{there exist positive c, $n_0$, and $m_0$ such that $0 \le f(n,m)  \le cg(n,m)$ for all $n \ge n_0$ or $m \ge m_0$}\\
     \end{cases}
\]

Give corresponding difenition for $\Omega(g(m,n))$ and  $\Theta(g(m,n))$ 

\[ \Omega(g(n,m)) = 
\begin{cases}
       \text{f(n,m),} &\quad\text{there exist positive c, $n_0$, and $m_0$ such that $cg(n,m) \le f(n,m) $ for all $n \ge n_0$ or $m \ge m_0$}\\
     \end{cases}
\]


\[ \Theta(g(n,m)) = 
\begin{cases}
       \text{f(n,m),} &\quad\text{there exist positive c, $n_0$, and $m_0$ such that $c_1g(n,m) \le f(n,m) \le c_2g(n,m) $ for all $n \ge n_0$ or $m \ge m_0$}\\
     \end{cases}
\]
\end{document}